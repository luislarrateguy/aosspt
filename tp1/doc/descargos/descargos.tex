\section{Descargos\label{sec:descargos}}

\subsection{Locales}
El enunciado dice que sistema debe funcionar en modo case insensitive. Si
observamos la tabla ASCII\footnote{http://www.asciitable.com/} podemos ver que
la diferencia entre las letras minúsculas y mayúsculas es siempre 0x20 (32 en
decimal). Por lo tanto es fácil pasar de mayúsculas a minúsculas, o viceversa.

El problema se presenta con las vocales acentuadas. Si bien nuestro código
funciona en nuestras máquinas (si buscamos una ``é'' o ``É'' devolverá ``San
Martin, José''), todo depende del locale que utilice el sistema donde se
ejecuta el cliente. El mismo debe ser ISO-8859-1 o ISO-8859-15. No habría
problemas con el servidor, ya que éste sólo lee el archivo de guía (cuya
codificación por supuesto es también ISO-8859-1), y no recibe datos del usuario
a través de la consola. Aunque como éste también arroja información de las
peticiones por consola, sería conveniente (pero no necesario para el
funcionamiento) cambiar la codificación de la consola para el servidor también.

En cambio sí es necesario cambiar la codificación de la terminal para el
cliente.  Si se utiliza la terminal de GNOME, es fácil desde el menú
\emph{Terminal} cambiar la codificación de la misma a ISO-8859-1 o ISO-8859-15.
Por ejemplo, si se utiliza UTF-8 en lugar de una de las mencionadas
anteriormente, las letras acentuadas constan de dos caracteres. De esta forma
el sistema no funciona: el cliente envía datos que el servidor espera con otra
codificación.

