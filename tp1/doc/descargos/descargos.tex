\section{Descargos}

\subsection{Locales}
El enunciado dice que sistema debe funcionar en modo case insensitive. Si
observamos la tabla ASCII\footnote{http://www.asciitable.com/} podemos ver que
la diferencia entre las letras minúsculas y mayúsculas es siempre 0x20 (32 en
decimal). Por lo tanto es fácil pasar de mayúsculas a minúsculas, o viceversa.

El problema se presenta con las vocales acentudadas. Si bien nuestro código
funciona en nuestras máquinas (si buscamos una ``é'' o ``É'' devolverá ``San
Martin, José''), todo depende del locale que utilice el sistema donde se
ejecuta el cliente. El mismo debe ser ISO-8859-1 o ISO-8859-15. No habría
problemas con el servidor, ya que éste sólo lee el archivo de guía (cuay
codificación por supuesto es también ISO-8859-1), y no recibe datos del usuario
a través de la consola.

Si se utiliza la terminal de Gnome, es fácil, desde el menú \emph{Terminal}
cambiar la codificación de la misma a ISO-8859-1 o ISO-8859-15.

