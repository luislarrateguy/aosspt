\subsection{El servidor}

\lgrindfile{src_tex/servidor.c.tex}{Archivo \archivo{servidor.c}}

% CÓDIGO DE COMUNICACIONES
\subsubsection{Código de comunicaciones}

En la línea 29, donde se llama a \metodo{setsockopt}, vale la pena una
aclaración. El método, como dice el comentario, sirve para setear algunas
opciones al socket. La que se setea allí es \variable{SO\_REUSEADDR}. En la
sección 7 del manual para \emph{socket} leemos:

\begin{quote}

\textbf{SO\_REUSEADDR}\linebreak
Indica que las reglas usadas para validar las direcciones proporcionadas en una
llamada bind(2) deben permitir la reutilización  de  las direcciones  locales.
Para  los conectores PF\_INET esto significa que un conector se puede enlazar a
una dirección, excepto cuando hay un conector activo escuchando asociado a la
dirección. Cuando el conector que está escuchando está asociado a INADDR\_ANY
con un puerto específico, entonces no es posible realizar enlaces a este puerto
para ninguna dirección local.

\end{quote}

También vale otra aclaración al observar la línea 34. El servidor almacena en
una cola las solicitudes que llegan cuando éste no está disponible para
atenderlas (o sea, cuando esta trabajando en otra solicitud). La constante
\variable{TAM\_COLA}, que esta definida en el archivo \archivo{datos.h}, es la
que indica el tamaño de la cola, que es 10.

% CODIGO DE LECTURA DE LA GUIA
\subsubsection{Código de lectura de la guía}

\TODO{Falta completar esta sección (sólo si hace falta comentar el código encargado
de la lectura de la guía). Encargado: César}
