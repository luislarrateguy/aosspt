\section{Principal}

% RTLinux
\subsection{Características de la máquina con RTLinux}

\subsubsection{Hardware}
\begin{itemize}
\item \textbf{CPU:} AMD Athlon XP 2166 MHz
\item \textbf{RAM:} 768 MB
\end{itemize}

\subsubsection{Software}
\begin{itemize}
\item \textbf{Sistema operativo:} Debian 3.0
\item \textbf{Versión de RTLinux:} 3.1
\end{itemize}


% Minix4RT
\subsection{Características de la máquina con Minix4RT}

\subsubsection{Hardware}
\begin{itemize}
\item \textbf{Modelo:} Notebook IBM ThinkPad 370C
\item \textbf{CPU:} 486
\item \textbf{RAM:} 8 MB
\end{itemize}

\subsubsection{Software}
\begin{itemize}
\item \textbf{Sistema operativo:} Minix 2.0.2
\end{itemize}


% Resultado de las mediciones
\subsection{Resultados completos de las mediciones}
Los resultados completos de las mediciones se pueden encontrar en la planilla
de OpenOffice Calc (o Excel, se incluyen ambos formatos).


% Tablas de resumen
\subsection{Tablas de resumen de los resultados}

\subsubsection{Medición de latencia de interrupciones del sistema Monitor}
\begin{center}
\begin{tabular}{|c|c|c|c|}
\hline
&\multicolumn{3}{|c|}{FRECUENCIAS}\\
\hline
&1000&2000&5000\\
\hline
MEDIA&999996,61&499998,66&199999,17\\
\hline
STD&1079,02&1702,82&2229,71\\
\hline
MIN&987424&486880&186624\\
\hline
MAX&1012000&513184&213152\\
\hline
DEADLINES&0&0&0\\
\hline
\end{tabular}
\end{center}


\subsubsection{Medición de puntualidad de MINIX4RT}

\begin{center}
\begin{tabular}{|c|c|c|c|}
\hline
&\multicolumn{3}{|c|}{EN VACÍO}\\
\hline
&1000&2000&5000\\
\hline
MEDIA&999102.53&499566.27&199488.06\\
\hline
STD&14372.60&2791.26&3314.72\\
\hline
MIN&771840&485440&178144\\
\hline
MAX&1218144&518048&212544\\
\hline
DEADLINES&0&0&0\\
\hline
\end{tabular}
\end{center}

\begin{center}
\begin{tabular}{|c|c|c|c|}
\hline
&\multicolumn{3}{|c|}{CON CARGA}\\
\hline
&1000&2000&5000\\
\hline
MEDIA&999122.11&499548.29&199408.83\\
\hline
STD&10896.38&12700.67&5012.85\\
\hline
MIN&862976&311456&157696\\
\hline
MAX&1136064&688192&236320\\
\hline
DEADLINES&0&0&0\\
\hline

\end{tabular}
\end{center}


% Gráficas
\subsection{Gráficos de los resultados}

Hemos hecho los histogramas con los datos de las mediciones locales en RTLinux,
y con y sin carga en Minix4RT.

\incluirimagen{width=\textwidth}{principal/graficas/local1000.pdf}
\incluirimagen{width=\textwidth}{principal/graficas/local2000.pdf}
\incluirimagen{width=\textwidth}{principal/graficas/local5000.pdf}
\incluirimagen{width=\textwidth}{principal/graficas/sc1000.pdf}
\incluirimagen{width=\textwidth}{principal/graficas/sc2000.pdf}
\incluirimagen{width=\textwidth}{principal/graficas/sc5000.pdf}
\incluirimagen{width=\textwidth}{principal/graficas/cc1000.pdf}
\incluirimagen{width=\textwidth}{principal/graficas/cc2000.pdf}
\incluirimagen{width=\textwidth}{principal/graficas/cc5000.pdf}

\pagebreak
% Conclusiones
\subsection{Conclusiones}

Observando las tablas con los resultados de Minix4RT, vemos que en los 1000 Hz
parece comportarse mejor con carga que sin ella. Pero en los 2000 Hz se puede
ver cómo la desviación estándar es menor en la corrida sin carga, y los mínimos
y máximos se acercan más a los 0,5 milisegundos. Lo mismo pasa en los 5000 Hz:
la desviación estándar es menor, y los mínimos y máximos se acercan más a los
0,2 milisegundos en la corrida sin carga que en la otra.

Los gráficos dicen lo mismo: observando, por ejemplo, el gráfico de Minix4RT
sin carga a 2000 Hz, y comparándolo con el de la corrida con carga a la misma
frecuencia, podemos observar cómo en el primero las muestras se concentran más
cerca del medio milisegundo, como debe ser, y en el otro están un poco más
dispersas.

Este comportamiento es el esperado: Sin carga funciona mejor que con ella.

