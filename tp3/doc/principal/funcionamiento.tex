\subsection{Descripción del funcionamiento}

% mutexd
\subsubsection{\proceso{mutexd}}

Un proceso \proceso{mutexd} se ejecuta de la siguiente forma:

\comando{\$ ./mutexd numeroPuerto}

\emph{numeroPuerto} indica el puerto donde el servidor va a escuchar. También
utiliza este valor para buscar en el archivo de configuración
\archivo{/etc/mutex.cfg} quién será su \emph{holder}. No sólo examina el
registro del archivo con el valor del puerto pasado como argumento, sino que
analiza todos los registros para conocer quiénes son los demás servidores.

Cuando termina esta tarea de lectura del archivo de configuración, procede a
enviar un mensaje \emph{HELLO} a cada uno de los servidores. Cada servidor que
recibe un mensaje \emph{HELLO} marca al emisor del mismo como activo. Ningún
servidor comienza su algoritmo hasta que no tenga marcados a todos los demás
servidores como activos.


% /etc/mutex.cfg
\subsubsection{Archivo de configuración \archivo{/etc/mutex.cfg}}

El archivo de configuración tiene la siguiente estructura:

\salida{../etc/mutex.cfg}

\FIXME{Cuando lo encuentra, revisa si el valor correspondiente en la segunda columna
es un número natural, correspondiente a un puerto, y de ser así sabe que a
traves de ese puerto puede comunicarse con su \emph{holder}.
Directamente al holder?. A self y holder les asigné el puerto, cosa e enviar directo
el mensaje. Recuerden que solo debe enviarse al vecino}

Cuando un proceso \proceso{mutexd} lee el archivo de configuración en busca de
su \emph{holder}, utilizando \emph{numeroPuerto}, lo hace revisando la primer
columna del mismo\footnote{Las columnas se separan con un espacio en blanco.}.
Cuando lo encuentra, revisa si el valor correspondiente en la segunda columna
es un número natural, correspondiente a un puerto, y de ser así sabe que a
traves de ese puerto puede comunicarse con su \emph{holder}, y también sabe que
él no tiene el token actualmente. En cambio si el valor de la segunda columna es
0 (cero), entonces ese servidor tendrá el token.


% mutexloop
\subsubsection{\proceso{mutexloop}}

Un proceso \proceso{mutexloop} se ejecuta de la siguiente forma:

\comando{\$ ./mutexloop puertoServidor}

\emph{puertoServidor} le indica en qué puerto debe conectarse para pedirle a su
proceso \proceso{mutexd} correspondiente el permiso para entrar a la región
crítica. Luego simplemente hace lo que tiene que hacer y sale de la misma,
informándoselo al servidor.


% Protocolo
\subsubsection{Protocolo de comunicación}

Los mensajes de comunicación son los siguientes:

\TODO{Revisar si los mensajes son correctos.  Revisé. Cambié un nombre
de acuerdo al código. Elijan el que les parezca más adecuado.}
\begin{itemize}

\item \textbf{REQUEST}: Es utilizado tanto por los servidores para el
intercambio del token, como por los clientes hacia los servidores. Indica que
el emisor desea obtener el token, para acceder (cliente) o dar permiso de
acceder (servidor) a la región crítica.

Los servidores lo envían a su \emph{HOLDER} sólo en caso de no poseer el token,
y los clientes lo envían a su correspondiente servidor cuando intentan acceder
la región crítica.

\item \textbf{PRIVILEGE}: Sólamente puede enviarlo un servidor que tenga el
token, pero el destinatario puede ser un cliente como otro servidor. La
transferencia de un servidor \textsl{i} a uno \textsl{j} de este mensaje
implica, primero, que el servidor \textsl{i} posee actualmente el token.
Segundo, que el servidor \textsl{i} es hasta el momento el \emph{HOLDER} del
servidor \textsl{j}. Tercero, que luego de recibido el mensaje, el servidor
\textsl{j} será el que actualmente tenga el token, y cuarto, el \emph{HOLDER}
del servidor \textsl{i} será \textsl{j}.

\item \textbf{SALIR\_RC}: Es utilizado únicamente por el cliente, y esta siempre
dirigido a su correspondiente servidor. Sirve para indicarle que ha terminado
de trabajar sobre la región crítica y desea liberar.

\item \textbf{ENTRAR\_RC}: Unicamente utilizado por el clietne para pedir
acceso exclusivo a la región crítica.

\end{itemize}

