\subsection{Descripción del funcionamiento}

Nos hemos guiado del documento \emph{A Tree-Based Algorithm for Distributed
Mutual Exception}, de Kerry Raymond, University of Queensland.

% mutexd
\subsubsection{\proceso{mutexd}}

Un proceso \proceso{mutexd} se ejecuta de la siguiente forma:

\comando{\$ ./mutexd numeroPuerto}

\emph{numeroPuerto} indica el puerto donde el servidor va a escuchar. También
utiliza este valor para buscar en el archivo de configuración
\archivo{/etc/mutex.cfg} quién será su \emph{holder}. No sólo examina el
registro del archivo con el valor del puerto pasado como argumento, sino que
analiza todos los registros para conocer quiénes son los demás servidores.

Cuando termina esta tarea de lectura del archivo de configuración, procede a
enviar un mensaje \emph{HELLO} a cada uno de los servidores. Cada servidor que
recibe un mensaje \emph{HELLO} marca al emisor del mismo como activo. Ningún
servidor comienza su algoritmo hasta que no tenga marcados a todos los demás
servidores como activos.

Por cuestiones de simplificación de la implementación, y dado que los servidores
van a correr en la misma máquina, cada uno es identificado por su numero de
puerto. Es decir que cuando son encolados como interesados en recibir el 
privilegio, lo que se agrega en la cola es el numero de puerto que utilizan para
recibir los mensajes. Esto podría ser modificado para que utilice algún código
hash de identificación o bien algún otro mecanismo que permita lograr el fin. 
Ej: hash("ip:puerto")


% /etc/mutex.cfg
\subsubsection{Archivo de configuración \archivo{/etc/mutex.cfg}}

El archivo de configuración tiene la siguiente estructura:

\salida{../etc/mutex.cfg}

Cuando un proceso \proceso{mutexd} lee el archivo de configuración en busca de
su \emph{holder}, utilizando \emph{numeroPuerto}, lo hace revisando la primer
columna del mismo\footnote{Las columnas se separan con un espacio en blanco.}.
Cuando lo encuentra, revisa si el valor correspondiente en la segunda columna
es un número natural, correspondiente a un puerto, y de ser así sabe que a
través de ese puerto puede comunicarse con su \emph{holder}, y también sabe que
él no tiene el token actualmente. En cambio si el valor de la segunda columna es
0 (cero), entonces ese servidor tendrá el token.

En el archivo mostrado arriba, el servidor que escuchará en 5001 posee el token.
El 5002 tiene como \emph{holder} a 5001, etc.

Por defecto nuestro código busca el archivo \archivo{mutex.cfg} en
\archivo{../etc/} y no en \archivo{/etc/}. Esto es así porque el formato del
mismo no es igual al que se propone en el enunciado. Si se desea cambiarlo
simplemente hay que abrir el archivo \archivo{src/mutexd.c} y cambiar la
constante \variable{RUTA\_MUTEX\_CFG} por el valor deseado.


% mutexloop
\subsubsection{\proceso{mutexloop}}

Un proceso \proceso{mutexloop} se ejecuta de la siguiente forma:

\comando{\$ ./mutexloop puertoServidor}

\emph{puertoServidor} le indica en qué puerto debe conectarse para pedirle a su
proceso \proceso{mutexd} correspondiente el permiso para entrar a la región
crítica. Luego simplemente hace lo que tiene que hacer y sale de la misma,
informándolo al servidor.

Un proceso \proceso{mutexloop} también escucha en un puerto. Cuando éste se
comunica con el servidor, coloca en uno de los campos de la estructura del
mensaje enviado el puerto que ha escogido para esperar una respuesta. El
mismo corresponde a la siguiente fórmula: $puertoServidor + 1000$. O sea que
mientras que los servidores escuchan en los puertos 5001\ldots5007, los clientes
lo hacen en el rango 6001\ldots6007. Por supuesto que estos dos rangos deben
estar libres en la máquina de pruebas. La decisión de utilizar de esta forma el 
puerto escucha en el cliente tiene el fin de tratar los mensajes de una forma
uniforme en el servidor. Ver más detalles en el protocolo de comunicación.


% Protocolo
\subsubsection{Protocolo de comunicación}

Un mensaje de comunicación envía una estructura llamada \variable{msg}, la cual
incluye un TIPO de mensaje (definido debajo) y un FROM que indica el puerto a
donde el emisor espera una respuesta. Este puerto permite identificar quien está
enviando el mensaje. Ver su implementación en el código.

Los mensajes de comunicación son los siguientes:

\begin{itemize}

\item \textbf{HELLO}: Mensaje utilizado en la inicialización entre servidores.
Cuando un servidor comienza, envía un mensaje de éste tipo a todos los demás
servidores, y luego espera las respuestas de todos ellos. El algoritmo comienza
solamente si el servidor sabe que todos los demás servidores están activos.

Al ejecutarse los procesos \proceso{mutexd}, se puede ver a medida que se
corren los mismos, cómo éstos se van avisando entre ellos que están activos.
Sin embargo el último en ejecutarse enviará un mensaje \emph{HELLO} de más a
los otros procesos. Esto es así porque al recibir un mensaje \emph{HELLO} de un
emisor, el receptor contesta con otro mensaje \emph{HELLO} (Ver más detalles en
la sección \ref{sec:servidor}, función \metodo{saludo\_inicial}). Este es un
detalle simplemente, el algoritmo de inicialización funciona correctamente.

\item \textbf{REQUEST}: Es utilizado por los servidores para pedir por el token
a su \emph{holder}. Indica que el proceso \proceso{mutexd} emisor desea obtener
el token para dar permiso de acceder a la región crítica a su cliente.

Los servidores lo envían a su \emph{HOLDER} sólo en caso de no poseer el token.

\item \textbf{PRIVILEGE}: Solamente puede enviarlo un servidor que tenga el
token, pero el destinatario puede ser un cliente como otro servidor. La
transferencia de un servidor \textsl{i} a uno \textsl{j} de este mensaje
implica, primero, que el servidor \textsl{i} posee actualmente el token.
Segundo, que el servidor \textsl{i} es hasta el momento el \emph{HOLDER} del
servidor \textsl{j}. Tercero, que luego de recibido el mensaje, el servidor
\textsl{j} será el que actualmente tenga el token, y cuarto, el \emph{HOLDER}
del servidor \textsl{i} será \textsl{j}.

\item \textbf{ENTRAR\_RC}: Únicamente utilizado por el cliente para pedir
acceso exclusivo a la región crítica. El efecto de este mensaje es hacer que el 
propio servidor \emph{mutexd} se encole como interesado en el \emph{TOKEN}.

\item \textbf{SALIR\_RC}: Es utilizado únicamente por el cliente, y está
siempre dirigido a su correspondiente servidor. Sirve para indicarle que ha
terminado de trabajar sobre la región crítica y desea liberarla.

\end{itemize}

